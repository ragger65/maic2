\documentclass[12pt,a4paper]{article}

\usepackage[DIV13,BCOR0mm]{typearea}
\usepackage{setspace}

\usepackage{natbib}
\bibpunct[, ]{(}{)}{,}{a}{}{,}

% \usepackage{wasysym}

% Check if we are compiling under latex or pdflatex
% \ifx\pdftexversion\undefined
%  \usepackage[dvips]{graphicx}
% \else
   \usepackage[pdftex]{graphicx}
% \fi

% \graphicspath{{../Figs/}}

\usepackage{hyperref}
\usepackage{hyperxmp}
\hypersetup{
    pdfauthor={Ralf Greve, Bj\"orn Grieger, Oliver J. Stenzel},
    pdftitle={MAIC-2 - Quick Start Manual},
    pdfsubject={Martian climate model},
    pdfkeywords={Mars, Planetary ice, Obliquity, %
                 Ice cap, Polar layered deposits, Ice age},
    pdfcopyright={GNU General Public License version 3 or later},
    pdflicenseurl={http://www.gnu.org/licenses/}}

% Positioning of figures
% (values chosen such that figure is inserted at the place of the call
% if possible):
\setcounter{topnumber}{10}
\setcounter{bottomnumber}{10}
\setcounter{totalnumber}{20}
\renewcommand{\topfraction}{0.99}
\renewcommand{\textfraction}{0.01}
\renewcommand{\bottomfraction}{0.99}

\onehalfspacing

\input{newcommands.tex}

%%%%%%%%%%%%%%%%%%%%%%%%%%%%%%%%%%%%%%%%%%%%%%%%%%%%%%%%%%%%%%%%%%%%%%%%

\begin{document}

\begin{center}

\rule{0mm}{15mm}

\begin{huge}
  \textbf{MAIC-2\\[-0.5ex]
          -- Quick Start Manual --}\\[-0.5ex]
\end{huge}

\rule{0mm}{25mm}

\begin{large}
  \textsc{Ralf Greve}
\end{large}

\vspace*{2mm}

\begin{normalsize}
  Institute of Low Temperature Science, Hokkaido University,\\[-0.25ex]
  Kita-19, Nishi-8, Kita-ku, Sapporo 060-0819, Japan\\[1ex]
\end{normalsize}

\vspace*{6mm}

\begin{large}
  \textsc{Bj\"orn Grieger}
\end{large}

\vspace*{2mm}

\begin{normalsize}
  European Space Astronomy Centre (ESAC),\\[-0.25ex]
  Camino Bajo del Castillo s/n,\\[-0.25ex]
  28692 Villanueva de la Ca\~nada, Madrid, Spain\\[1ex]
\end{normalsize}

\vspace*{6mm}

\begin{large}
  \textsc{Oliver J.~Stenzel}
\end{large}

\vspace*{2mm}

\begin{normalsize}
  Max Planck Institute for Solar System Research,\\[-0.25ex]
  Justus-von-Liebig-Weg 3, 37077 G\"ottingen, Germany\\[1ex]
\end{normalsize}

\vfill

\begin{normalsize}\today\end{normalsize}
% \begin{normalsize}February 1, 2013\\
%                   (updated March 8, 2020)\end{normalsize}

\end{center}

\rule{0mm}{20mm}

\clearpage

\rule{0mm}{0mm}

\vfill

\begin{center}\begin{minipage}{0.85\textwidth}

\begin{small}

\noindent{}Copyright 2010-2026
           Ralf Greve, Bj\"orn Grieger, Oliver J.~Stenzel

\vspace*{1.5ex}

\noindent{}This file is part of MAIC-2.

\vspace*{1.5ex}

\noindent{}MAIC-2 is free software. It can be redistributed and/or modified under the terms of the GNU General Public License (http://www.gnu.org/licenses/) as published by the Free Software Foundation, either version~3 of the License, or (at the user's option) any later version.

\vspace*{1.5ex}

\noindent{}MAIC-2 is distributed in the hope that it will be useful, but WITHOUT ANY WARRANTY; without even the implied warranty of MERCHANTABILITY or FITNESS FOR A PARTICULAR PURPOSE.  See the GNU General Public License for more details.

\end{small}

\end{minipage}\end{center}

\vfill

\rule{0mm}{0mm}

\clearpage

\section{Requirements}

\begin{itemize}

\item UNIX/LINUX system.

\item Fortran 90/95 compiler.

\end{itemize}

\section{Installation}

\begin{enumerate}

\item Clone the latest revision from the GitHub repository:

\hspace*{10mm}\verb+git clone https://github.com/ragger65/maic2.git+

\item You should then have a new folder ``maic2'' that contains the entire program package.

\end{enumerate}

\section{Files and directories in ``maic2''}

\begin{itemize}
\item \textbf{Main directory}:

Shell script (bash) maic2.sh for running a single simulation under UNIX/LINUX.

Shell script (bash) multi\_maic2.sh for running multiple simulations by repeated calls of maic2.sh.

Subdirectory \textbf{headers/templates}: specification files maic2\_specs\_\emph{run\_name}.h
\\
\hspace*{15.9em}(\emph{run\_name}: name of run).

\begin{center}\begin{tabular}{ll} \hline
  Name of Run & Description \\ \hline
  run\_c01a & Simulation \#2 by \citet{greve_etal_2010},\\[-0.35ex]
            & only over 1 Martian year with more detailed output \\
  run\_c01  & Simulation \#2 by \citet{greve_etal_2010} \\
  run\_c02  & Simulation \#1 by \citet{greve_etal_2010} \\
  run\_c03  & Simulation \#3 by \citet{greve_etal_2010} \\
  run\_c04  & Simulation \#4 by \citet{greve_etal_2010} \\ \hline
  run\_t06  & Simulation \#6 by \citet{greve_etal_2010} \\
  run\_t07  & Simulation \#7 by \citet{greve_etal_2010} \\
  run\_t08  & Simulation \#8 by \citet{greve_etal_2010} \\
  run\_t12  & Simulation \#5 by \citet{greve_etal_2010} \\
  run\_t14  & Simulation \#6 by \citet{greve_etal_2010},\\[-0.35ex]
            & but from 20~Ma ago until 10~Ma into the future \\ \hline
  run\_t39  & Simulation by \citet{greve_etal_2012} (pp.\,14-15) \\
  run\_t40  & Simulation by \citet{greve_etal_2012} (pp.\,14-15) \\ \hline
\end{tabular}\end{center}

Copy these header files into the subdirectory \textbf{headers} by executing

\hspace*{10mm}\verb+cp headers/templates/maic2_specs_*.h headers/+

\item \textbf{src}:

Main program file maic2.F90.

Subdirectory \textbf{subroutines}: subroutines for MAIC-2.

\item \textbf{maic2\_in}:

Input data files (orbital forcing) for MAIC-2.

\item \textbf{maic2\_out}:

Default directory for output files produced by MAIC-2; initially empty.

\item \textbf{docs}:

Documentation for MAIC-2 (this Quick Start Manual, FORD developer manual). The latter can be built by executing \verb+ ford ford.md + in the main directory.

\end{itemize}

\section{How to run a simulation}

\begin{enumerate}

\item In the script maic2.sh, search for ``greve'', and replace the path names for RUN\_DIR and SRC\_DIR with your own ones.

Also, search for ``Compiler'', and replace the variables F90 and F90FLAGS according to the syntax of your own Fortran compiler (F90FLAGS should do).

\item In the specification files (subdirectory headers/), search for ``greve'', and replace the path names for INPATH and OUTPATH with your own ones.

\item The rest is quite simple:

\begin{itemize}

\item In order to run simulation run\_t06, use the script maic2.sh. The command is
\\
\hspace*{10mm}\verb+(./maic2.sh run_t06) >tmp/out_job.dat 2>&1 &+
\\
(bash required). Accordingly for the other simulations.

\item Alternatively, if you prefer to run all simulations consecutively, you may use the script multi\_maic2.sh:
\\
\hspace*{10mm}\verb+(./multi_maic2.sh) >tmp/out_mjob.dat 2>&1 &+

\end{itemize}

\end{enumerate}

\noindent{}The computing times for the simulations, run with the Intel Fortran Compiler for Linux 11.1 (optimization option --fast) on an Intel Xeon X5570 (2.93~GHz) PC under openSUSE 11.0 (64~bit), are as follows:

\begin{center}\begin{tabular}{lr@{.}lcclr@{.}lcclr@{.}l} \hline
  Run & \multicolumn{2}{c}{Time} &&&
  Run & \multicolumn{2}{c}{Time} &&&
  Run & \multicolumn{2}{c}{Time} \\ \hline
  run\_c01a\qquad &  0&$1\,\mathrm{sec}$ &&&
  run\_t06\qquad  &  7&$0\,\mathrm{hrs}$ &&&
  run\_t39\qquad  &  7&$0\,\mathrm{hrs}$ \\
  run\_c01        &  7&$0\,\mathrm{hrs}$ &&&
  run\_t07        &  7&$0\,\mathrm{hrs}$ &&&
  run\_t40        & 14&$0\,\mathrm{hrs}$ \\
  run\_c02        &  7&$0\,\mathrm{hrs}$ &&&
  run\_t08        &  7&$0\,\mathrm{hrs}$ \\
  run\_c03        &  7&$0\,\mathrm{hrs}$ &&&
  run\_t12        &  7&$0\,\mathrm{hrs}$ \\
  run\_c04        &  7&$0\,\mathrm{hrs}$ &&&
  run\_t14        & 21&$0\,\mathrm{hrs}$ \\ \hline
\end{tabular}\end{center}

\section{Output files}
\label{sect_output}

Output files of simulations are written to a directory specified by the user (OUTPATH in specification files, see above). Each simulation produces an output file \textbf{run\_name.out} in ASCII format that contains the following data:

\begin{tabbing}
  \hspace*{3em} \= \hspace*{6em} \= \kill
  \> Column 1: \>Time $t$ [a] \\
  \> Column 2: \>Solar longitude $L_\mathrm{s}$ [deg] \\
  \> Column 3: \>Latitude $\varphi$ [deg] \\
  \> Column 4: \>Surface temperature $T(\varphi,t)$ [K] \\
  \> Column 5: \>Evaporation rate $E(\varphi,t)$ $[\mathrm{kg\,m^{-2}\,a^{-1}}]$ \\
  \> Column 6: \>Condensation rate $C(\varphi,t)$ $[\mathrm{kg\,m^{-2}\,a^{-1}}]$ \\
  \> Column 7: \>Water content $\omega(\varphi,t)$ $[\mathrm{kg\,m^{-2}}]$ \\
  \> Column 8: \>Net mass balance $a_\mathrm{net}(\varphi,t)$
                  $[\mathrm{mm\,a^{-1}}\mbox{ ice equivalent}]$ \\
  \> Column 9: \>Ice thickness $H(\varphi,t)$ [m] \\
\end{tabbing}

% \clearpage

% \section*{References}

\bibliographystyle{references_maic2}
\bibliography{references_maic2}

\end{document}
